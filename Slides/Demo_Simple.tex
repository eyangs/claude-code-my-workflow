% ==============================================================================
% Demo Lecture: Causal Inference Basics
% ==============================================================================
% 这是一个最小可运行的 Beamer 示例,用于演示完整工作流
% ==============================================================================

\documentclass[aspectratio=169, 12pt]{beamer}

% --- 引入头部文件 ---
% ==============================================================================
% Beamer Header - Minimal Template
% ==============================================================================
% 包含常用的 Beamer 主题设置和宏包
% ==============================================================================

% --- 语言与字体 ---
% \usepackage{ctex}  % 中文支持(需要额外安装)
\usepackage[utf8]{inputenc}
\usepackage[T1]{fontenc}
\usepackage{lmodern}

% --- 数学宏包 ---
\usepackage{amsmath, amssymb, amsthm}
\usepackage{mathtools}
\usepackage{bm}  % 粗体数学符号

% --- 图形与绘图 ---
\usepackage{graphicx}
\usepackage{tikz}
\usetikzlibrary{arrows, positioning, shapes}
\usepackage{booktabs}  % 表格线

% --- 主题设置 ---
\usetheme{Madrid}
\usecolortheme{beaver}

% --- 自定义环境 ---
\newenvironment{keybox}{
  \setbeamercolor{block title}{bg=orange!20,fg=black}
  \begin{block}}
  {\end{block}}

\newenvironment{definitionbox}[1]{
  \begin{block}{#1}}
  {\end{block}}


% --- 元数据 ---
\title{Causal Inference Basics}
\subtitle{因果推断基础入门}
\author{Demo Author}
\institute{Demo Institution}
\date{\today}

% ==============================================================================
% 文档开始
% ==============================================================================
\begin{document}

% --- 标题页 ---
\begin{frame}
  \titlepage
\end{frame}

% --- 大纲 ---
\begin{frame}{Outline}
  \tableofcontents
\end{frame}

% ==============================================================================
% Section 1: Introduction
% ==============================================================================
\section{Introduction}

\begin{frame}{What is Causal Inference?}
  \begin{definitionbox}{Definition}
    Causal inference is the process of drawing conclusions about causal relationships based on observed data \cite{Angrist2009_mostly_harmless}.
  \end{definitionbox}

  \vspace{0.5cm}

  \begin{keybox}{Key Question}
    What would have happened to Y if we had changed X by one unit, while holding everything else constant?
  \end{keybox}
\end{frame}

\begin{frame}{Why Causal Inference Matters}
  \begin{itemize}
    \item \textbf{Policy Evaluation}: Understanding the effect of interventions
    \item \textbf{Scientific Discovery}: Identifying causal mechanisms
    \item \textbf{Decision Making}: Making informed choices under uncertainty
  \end{itemize}

  \vspace{0.5cm}

  \textbf{Example:} Does taking a drug actually cure the disease, or is it just correlation?
\end{frame}

% ==============================================================================
% Section 2: Core Concepts
% ==============================================================================
\section{Core Concepts}

\begin{frame}{Potential Outcomes Framework}
  Let $Y_{1i}$ and $Y_{0i}$ denote the potential outcomes for unit $i$:

  \begin{align*}
    Y_{1i} &= \text{Outcome if treated} \\
    Y_{0i} &= \text{Outcome if not treated}
  \end{align*}

  The causal effect for unit $i$ is:
  \begin{equation}
    \tau_i = Y_{1i} - Y_{0i}
  \end{equation}

  \vspace{0.3cm}
  \textit{Problem: We can only observe one of these for each unit!}
\end{frame}

\begin{frame}{Average Treatment Effect (ATE)}
  The average causal effect across the population:
  \begin{equation}
    \text{ATE} = \mathbb{E}[Y_{1i} - Y_{0i}]
  \end{equation}

  \vspace{0.5cm}

  \begin{keybox}{Fundamental Problem of Causal Inference}
    We cannot observe $Y_{1i}$ and $Y_{0i}$ simultaneously for the same unit $i$.
  \end{keybox}
\end{frame}

% ==============================================================================
% Section 3: References
% ==============================================================================
\section{References}

\begin{frame}[allowframebreaks]{References}
  \bibliographystyle{plain}
  \bibliography{../Bibliography_base}
\end{frame}

% --- 结束页 ---
\begin{frame}
  \centering
  \Huge Thank You!

  \vspace{1cm}
  \Large Questions?
\end{frame}

\end{document}
% ==============================================================================
% 文档结束
% ==============================================================================
